% Prof. Dr. Ausberto S. Castro Vera
% UENF - CCT - LCMAT - Curso de Ci\^{e}ncia da Computa\c{c}\~{a}o
% Campos, RJ,  2024 
% Disciplina: An\'{a}lise e Projeto de Sistemas
% Aluno: Eric Hoffmann Fernandes Braga

\chapterimage{planejamento.png} % Table of contents heading image
\chapter{Etapa de Planejamento}


Neste capítulo será apresentado o planejamento do sistema, desde a requisição à execução e orçamento.

\section{Solicitação do Sistema}

\textbf{Responsável:} Eric Hoffmann Fernandes Braga \\
\textbf{Patrocinador:} Corridor Digital

\subsection{Necessidade de Negócio}

A \textbf{Corridor Digital}, renomada por suas produções criativas no YouTube e por liderar a indústria de efeitos visuais (VFX) e conteúdo digital inovador, busca expandir suas operações além de seu canal. Com uma base sólida em vídeos de curta-metragem, efeitos visuais impressionantes e um público engajado, a empresa deseja se transformar em um estúdio completo de produção audiovisual, abrangendo todas as etapas do processo criativo, desde a filmagem até a edição e pós-produção.

O objetivo da expansão é diversificar seu portfólio para incluir produções mais robustas, como filmes, séries, comerciais e conteúdos interativos voltados ao mercado de jogos e novas mídias. A meta é capitalizar seu \textit{know-how} em efeitos visuais, tecnologia emergente e \textit{storytelling} criativo, enquanto adota uma estrutura de estúdio que permita realizar projetos de maior escala e complexidade.

Para atingir esses objetivos, a \textbf{Corridor Digital} necessita de um sistema integrado e modular que otimize o gerenciamento de projetos, recursos, fluxos de trabalho criativos e técnicos, mantendo a flexibilidade e a inovação que são marcas registradas da empresa. Esse sistema precisa suportar novas capacidades como captura de movimento, produção em LED, realidade aumentada e virtual, e renderização em tempo real, maximizando a eficiência da equipe e ampliando a capacidade de entrega no competitivo mercado audiovisual.

\subsection{Requisitos de Negócio}

\begin{itemize}
    \item \textbf{Integração Total e Modularidade:} O sistema deve integrar todos os aspectos da produção audiovisual, desde a captura de imagens, edição, pós-produção, até a distribuição do conteúdo, oferecendo uma plataforma centralizada que facilita o fluxo de trabalho entre equipes criativas, técnicas e administrativas. A modularidade do sistema deve permitir a personalização para diferentes tipos de projetos e formatos.
    
    \item \textbf{Desempenho e Eficiência:} O sistema precisa processar grandes volumes de dados de vídeo e áudio com alta performance, minimizando latência e garantindo a eficiência em operações como renderização em tempo real, efeitos visuais (VFX) e outras tarefas intensivas de computação. O foco deve ser a redução de tempo de processamento para otimizar prazos de entrega.
    
    \item \textbf{Escalabilidade e Flexibilidade:} O sistema deve ser escalável para acomodar o crescimento futuro da empresa, suportando a adição de novas tecnologias, ferramentas criativas e técnicas, além de expansões na capacidade de produção. Ele deve ser flexível o suficiente para incorporar rapidamente inovações como realidade virtual, aumentada, captura de movimento e produção em LED.
    
    \item \textbf{Colaboração e Conectividade:} Deve permitir a colaboração em tempo real entre equipes, com integração de ferramentas colaborativas que facilitem o compartilhamento de arquivos, comunicação e revisão de projetos de forma fluida, mesmo entre equipes remotas ou em diferentes fusos horários.
    
    \item \textbf{Segurança e Proteção de Dados:} Implementar uma infraestrutura robusta de segurança para proteger dados sensíveis e materiais de produção, incluindo criptografia de dados, controle de acesso baseado em funções e backups regulares, garantindo a continuidade do negócio em caso de falhas. Além disso, deve atender a padrões de conformidade e privacidade para clientes e parceiros.
\end{itemize}

\subsection{Vantagens que o Sistema trará para a Organização}

\begin{itemize}
    \item \textbf{Aumento da Capacidade de Produção:} A nova infraestrutura permitirá à \textbf{Corridor Digital} expandir suas operações, otimizando o fluxo de trabalho e possibilitando a produção de um volume maior de conteúdo em diversos formatos, incluindo filmes, séries, comerciais e conteúdo digital, sem comprometer a eficiência.
    
    \item \textbf{Melhoria na Qualidade das Produções:} Com o uso de tecnologias avançadas, como captura de movimento, telas de LED e câmeras virtuais, a empresa será capaz de elevar o padrão de suas produções, resultando em um conteúdo visualmente mais sofisticado, capaz de atender às expectativas de um público cada vez mais exigente.
    
    \item \textbf{Exploração de Novos Projetos:} A infraestrutura aprimorada permitirá à \textbf{Corridor Digital} explorar novos formatos e gêneros, abrindo oportunidades para atender a uma base mais ampla de clientes, como estúdios de cinema, desenvolvedores de jogos e agências de publicidade, o que pode impulsionar o crescimento da empresa e aumentar suas fontes de receita.
    
    \item \textbf{Reforço da Reputação como Líder em Inovação:} Ao adotar e integrar tecnologias de ponta em suas operações, a \textbf{Corridor Digital} consolidará sua reputação como uma das empresas mais inovadoras na área de produção audiovisual, destacando-se no mercado por sua capacidade técnica e criatividade.
\end{itemize}

\subsection{Restrições}

\begin{itemize}
    \item \textbf{Gestão Financeira Moderada:} Embora o orçamento seja flexível, é essencial evitar gastos excessivos ou despesas imprevistas. O foco deve ser em otimizar os custos, garantindo que os recursos sejam utilizados de forma eficiente e estratégica, sem comprometer a qualidade ou o alcance do projeto.
    
    \item \textbf{Cronograma Flexível:} O projeto não está sujeito a prazos rígidos, já que as operações atuais da \textbf{Corridor Digital} não serão impactadas. No entanto, é importante manter uma gestão de tempo eficaz para evitar atrasos desnecessários e garantir que o sistema esteja operacional dentro de um prazo razoável.
    
    \item \textbf{Recursos Humanos Especializados:} A equipe da \textbf{Corridor Digital} já possui o conhecimento técnico necessário para operar as novas tecnologias do sistema, mas será necessária a contratação de novos membros dedicados para a gestão e manutenção do sistema. A integração e treinamento desses novos colaboradores devem ser realizados de forma organizada para garantir uma transição suave e eficiente.
    
    \item \textbf{Manutenção da Qualidade Operacional:} Durante a implementação, é fundamental garantir que as operações de produção existentes não sejam interrompidas. A integração do novo sistema deve ser feita de forma gradual e harmoniosa, garantindo que a qualidade e os prazos de produção atuais sejam mantidos.
\end{itemize}

%%%%%%%%%%%%%%%%%%%%%%%%%%%%%%%

\section{Custos: Desenvolvimento e Operacional}

\begin{itemize}
    \item \textbf{Custos de Desenvolvimento:} Estes custos englobam o investimento necessário na aquisição de novos equipamentos, software especializado e na contratação de consultores e desenvolvedores para garantir uma integração e personalização eficazes do sistema. A fase de desenvolvimento também pode incluir despesas com a configuração inicial e ajustes técnicos para atender às necessidades específicas da \textbf{Corridor Digital}.
    
    \item \textbf{Custos Operacionais:} Estes custos refletem as despesas contínuas relacionadas à manutenção e operação do sistema, incluindo atualizações de software, suporte técnico, e manutenção dos equipamentos. Adicionalmente, devem ser considerados os investimentos em treinamento contínuo da equipe para garantir que todos estejam atualizados com as melhores práticas e novas funcionalidades do sistema.
    
    \item Embora o projeto envolva despesas significativas, a \textbf{Corridor Digital} planeja mitigar parte desses custos por meio de contratos de patrocínio e parcerias estratégicas, que serão explorados em detalhes em uma seção subsequente.
\end{itemize}

%%%%%%%%%%%%%%%%%%%%%%%%%%%%%%%

\section{Benefícios}

\subsection{Benefícios Tangíveis}

\begin{itemize}
    \item \textbf{Aumento da Receita:} A expansão da capacidade de produção permitirá à \textbf{Corridor Digital} assumir novos projetos e atender a uma base de clientes mais ampla. Esta ampliação na capacidade de atendimento tem o potencial de gerar um aumento significativo na receita, ao diversificar e expandir as fontes de receita da empresa.
    
    \item \textbf{Redução de Custos Operacionais:} A integração e automação dos processos internos vão otimizar o fluxo de trabalho, resultando em uma redução significativa dos custos operacionais. A eliminação de tarefas manuais e a centralização dos sistemas de gestão contribuirão para uma maior eficiência e menores despesas operacionais.
    
    \item \textbf{Melhoria na Eficiência:} Com um sistema integrado, a \textbf{Corridor Digital} poderá melhorar a eficiência dos processos de produção e edição. A redução dos tempos de conclusão de projetos será possível através da automação e da otimização dos processos, resultando em uma produção mais ágil e eficiente.
\end{itemize}


\subsection{Benefícios Intangíveis}

\begin{itemize}
    \item \textbf{Fortalecimento da Imagem da Marca:} A adoção de tecnologias avançadas e a ampliação das capacidades de produção reforçarão a \textbf{Corridor Digital} como um líder inovador no setor audiovisual. A modernização e a inovação contribuirão para a percepção da empresa como uma referência em criatividade e tecnologia.
    
    \item \textbf{Satisfação do Cliente:} A melhoria na qualidade dos conteúdos e a capacidade de atender a uma gama mais ampla de necessidades contribuirão para um aumento na satisfação dos clientes. Um serviço de maior qualidade e a capacidade de atender a diversas demandas promoverão um feedback positivo e fortalecerão o relacionamento com os clientes.
    
    \item \textbf{Desenvolvimento da Equipe:} A introdução de novas tecnologias e o investimento em treinamento proporcionarão oportunidades de crescimento profissional para os funcionários. Este desenvolvimento contribuirá para a criação de uma cultura de inovação e aprimoramento contínuo dentro da empresa.
\end{itemize}

%%%%%%%%%%%%%%%%%%%%%%%%%%%%%%%

\section{Análise de custos e benefícios}

A análise de custos e benefícios avalia os investimentos necessários para o desenvolvimento e a operação do novo sistema em relação aos benefícios que ele proporcionará. Este processo abrange tanto os aspectos financeiros diretos quanto os impactos intangíveis, oferecendo uma visão abrangente do valor agregado à \textbf{Corridor Digital}.

\subsection{Investimentos Necessários}

Os principais investimentos incluem:

\begin{itemize}
    \item \textbf{Desenvolvimento e Implementação:} Os custos associados à aquisição de novos equipamentos, software especializado e à contratação de consultores e desenvolvedores para garantir a integração e personalização do sistema.
    \item \textbf{Custos Operacionais:} Despesas contínuas relacionadas à manutenção do sistema, atualizações de software, suporte técnico, operação dos equipamentos e treinamento da equipe.
\end{itemize}

\subsection{Impacto dos Patrocínios e Parcerias}

A \textbf{Corridor Digital} tem a oportunidade de aliviar parte dos custos através de patrocínios e parcerias estratégicas. A relevância e o alcance da empresa como um influente grupo de canais no YouTube podem atrair patrocinadores interessados em associar suas marcas ao novo sistema. As possíveis formas de patrocínio incluem:

\begin{itemize}
    \item \textbf{Patrocínios Financeiros:} Suporte financeiro direto para cobrir parte dos custos do projeto.
    \item \textbf{Fornecimento de Equipamentos e Software:} Parcerias que ofereçam equipamentos ou software como parte do acordo de patrocínio.
    \item \textbf{Colaborações em Campanhas Promocionais:} Parcerias para campanhas que beneficiem ambas as partes, aumentando a visibilidade e o alcance.
\end{itemize}

\subsection{Benefícios Esperados}

Os benefícios do novo sistema incluem a expansão da capacidade de produção, a redução de custos operacionais e a melhoria na eficiência dos processos. Além dos ganhos tangíveis, o sistema fortalecerá a imagem da marca, aumentará a satisfação dos clientes e contribuirá para o desenvolvimento profissional da equipe.

Essa análise fornece uma visão clara dos custos envolvidos e dos benefícios esperados, destacando a importância das parcerias e patrocínios para otimizar o impacto financeiro e garantir o sucesso do projeto.

%%%%%%%%%%%%%%%%%%%%%%%%%%%%%%%

\section{Estudo de Viabilidade}

\subsection{Viabilidade Técnica}
A viabilidade técnica do projeto é garantida pela disponibilidade de tecnologia avançada e pela infraestrutura existente da Corridor Digital. A nova infraestrutura proposta inclui:

\begin{itemize}
    \item \textbf{Estações de Edição:} Equipadas com processadores AMD Ryzen 9 7950X, 64 GB de RAM e placas gráficas NVIDIA RTX 4090. Essas estações atendem aos requisitos para renderização de alta qualidade e processamento intensivo de dados.
    \item \textbf{Servidores de Armazenamento:} Um servidor central com processador AMD EPYC 7742, 1 PB de armazenamento em RAID 6, 256 GB de RAM, e conectividade por fibra ótica, assegurando alta capacidade e velocidade de armazenamento.
    \item \textbf{Equipamentos de Motion Capture e Scanning 3D:} Tecnologias avançadas como câmeras LIDAR e trajes Rokoko são compatíveis com os requisitos do projeto, garantindo captura e processamento de dados de alta precisão.
    \item \textbf{Tela Infinita de LED:} Um painel de LED de alta resolução com 20m de largura e 5m de altura, instalado em formato circular, permitirá a criação de cenários virtuais imersivos em tempo real.
\end{itemize}

A equipe técnica da Corridor Digital possui experiência com estas tecnologias, e os novos equipamentos são compatíveis com as infraestruturas existentes, assegurando uma integração suave e eficiente.

\subsection{Viabilidade Organizacional}
A viabilidade organizacional do projeto é suportada pela estrutura e experiência da equipe da Corridor Digital. O projeto se beneficiará das seguintes condições:

\begin{itemize}
    \item \textbf{Experiência da Equipe:} A equipe atual da Corridor Digital já possui conhecimento e experiência com as tecnologias propostas. No entanto, serão contratados especialistas adicionais para gerenciar a nova infraestrutura e garantir uma integração eficiente.
    \item \textbf{Capacidade de Adaptação:} A estrutura organizacional existente está preparada para incorporar novas tecnologias sem comprometer as operações diárias. A organização tem um histórico de adaptação bem-sucedida a novas ferramentas e processos.
    \item \textbf{Treinamento e Suporte:} Embora a equipe já esteja familiarizada com as tecnologias, um programa de treinamento será implementado para assegurar a proficiência total com os novos sistemas e garantir que todos os membros da equipe estejam alinhados com as novas operações.
\end{itemize}

O planejamento cuidadoso e o suporte contínuo garantirão uma transição suave e eficaz para a nova infraestrutura.

\subsection{Viabilidade Econômica}
A viabilidade econômica do projeto considera os custos e benefícios associados à implementação da nova infraestrutura. A análise econômica inclui:

\begin{itemize}
    \item \textbf{Custos de Implementação:} O investimento inicial será destinado à aquisição dos equipamentos avançados e à contratação de novos profissionais especializados. Apesar do investimento significativo, o orçamento é flexível e bem planejado para evitar surpresas financeiras.
    \item \textbf{Benefícios Esperados:} O aumento da capacidade de produção e a eficiência aprimorada permitirão à Corridor Digital assumir novos projetos e clientes, potencialmente aumentando a receita. A redução de custos operacionais através da automação e integração dos processos contribuirá para a economia geral.
    \item \textbf{Apoio Financeiro:} Como a Corridor Digital possui uma base sólida de patrocinadores e oportunidades de parcerias, será possível obter apoio financeiro adicional para aliviar parte dos custos. A negociação de contratos de patrocínio ajudará a cobrir uma parte dos investimentos necessários.
\end{itemize}

A análise econômica sugere que os benefícios financeiros e operacionais superam os custos iniciais, tornando o projeto viável e justificável do ponto de vista econômico.
\subsection{Calendário}
O projeto será executado ao longo de dois anos, divididos em quatro semestres. Cada semestre terá metas específicas e marcos importantes para garantir o progresso contínuo e a conclusão bem-sucedida do projeto.

\begin{itemize}
    \item \textbf{Janeiro - Junho do Primeiro Ano: Preparação e Planejamento}
    \begin{itemize}
        \item Finalização dos requisitos e especificações do projeto.
        \item Aquisição e configuração inicial dos equipamentos e softwares.
        \item Formação da equipe e início do treinamento técnico.
    \end{itemize}
    \item \textbf{Julho - Dezembro do Primeiro Ano: Implementação Inicial}
    \begin{itemize}
        \item Instalação e configuração de estações de trabalho e servidores.
        \item Implementação dos sistemas de captura de movimento e scanning 3D.
        \item Desenvolvimento de interfaces de usuário e integração com o sistema.
    \end{itemize}
    \item \textbf{Janeiro - Junho do Segundo Ano: Testes e Ajustes}
    \begin{itemize}
        \item Testes de integração e validação dos sistemas implementados.
        \item Ajustes e otimizações com base no feedback dos testes.
        \item Realização de treinamento avançado e simulações de operações.
    \end{itemize}
    \item \textbf{Julho - Dezembro do Segundo Ano: Finalização e Lançamento}
    \begin{itemize}
        \item Revisão final e correção de problemas identificados.
        \item Preparação e realização do lançamento oficial do sistema.
        \item Avaliação do projeto e coleta de feedback para futuras melhorias.
    \end{itemize}
\end{itemize}

\subsection{Cronograma}
O cronograma detalhado será dividido em meses, com marcos e atividades específicas para cada fase do projeto. A seguir está uma visão geral do cronograma para garantir a conclusão bem-sucedida do projeto:


\begin{table}[H]
\centering
\begin{tabular}{|p{2cm}|p{4cm}|p{3cm}|p{3cm}|p{2cm}|}
\hline
\textbf{Mês} & \textbf{Atividades} & \textbf{Marcos} & \textbf{Responsáveis} & \textbf{Status} \\ \hline
Janeiro & Reuniões de planejamento, definição de requisitos & Documento de requisitos finalizado & Gerente de Projeto, Equipe Técnica & Em Andamento \\ \hline
Fevereiro & Aquisição de equipamentos, início do treinamento & Equipamentos adquiridos, Treinamento iniciado & Equipe de TI, RH & Em Andamento \\ \hline
Março & Configuração inicial dos sistemas & Sistemas configurados & Equipe Técnica & Em Andamento \\ \hline
Abril & Instalação de estações e servidores & Estações e servidores instalados & Equipe de TI & Planejado \\ \hline
Maio & Implementação dos sistemas de captura e scanning & Sistemas de captura e scanning em funcionamento & Equipe Técnica & Planejado \\ \hline
Junho & Desenvolvimento de interfaces e integração & Interfaces desenvolvidas e integradas & Equipe de Desenvolvimento & Planejado \\ \hline
Julho & Início dos testes de integração & Testes de integração iniciados & Equipe de Qualidade & Planejado \\ \hline
Agosto & Ajustes e otimizações & Sistema ajustado e otimizado & Equipe Técnica & Planejado \\ \hline
Setembro & Treinamento avançado, simulações & Treinamento concluído, simulações realizadas & Equipe de Treinamento & Planejado \\ \hline
Outubro & Revisão final do sistema & Sistema revisado e corrigido & Equipe Técnica & Planejado \\ \hline
Novembro & Preparação para o lançamento & Preparativos para o lançamento concluídos & Equipe de Marketing & Planejado \\ \hline
Dezembro & Lançamento oficial e avaliação do projeto & Lançamento realizado & Gerente de Projeto, Equipe Técnica & Planejado \\ \hline
Janeiro do Segundo Ano & Revisão pós-lançamento, ajustes finais & Ajustes pós-lançamento concluídos & Equipe Técnica & Planejado \\ \hline
Fevereiro & Coleta de feedback e planejamento de melhorias & Feedback coletado, plano de melhorias definido & Gerente de Projeto & Planejado \\ \hline
Março & Implementação de melhorias & Melhorias implementadas & Equipe Técnica & Planejado \\ \hline
Abril & Análise de desempenho e ajustes & Desempenho analisado, ajustes realizados & Equipe de Qualidade & Planejado \\ \hline
Maio & Preparação de documentação e relatórios finais & Documentação e relatórios finalizados & Gerente de Projeto & Planejado \\ \hline
Junho & Encerramento formal do projeto & Projeto encerrado formalmente & Gerente de Projeto, Equipe Técnica & Planejado \\ \hline
Julho - Dezembro do Segundo Ano & Suporte pós-implementação e manutenção & Suporte contínuo & Equipe de Suporte & Planejado \\ \hline
\end{tabular}
\caption{Cronograma do Projeto}
\label{tab:cronograma}
\end{table}



\subsection{Alternativas Tecnológicas}

\subsubsection{Software}

\paragraph{Software de Edição}
\begin{itemize}
    \item Adobe Premiere Pro
    \item DaVinci Resolve
    \item Software proprietário especializado
\end{itemize}

\paragraph{Software de Motion Capture}
\begin{itemize}
    \item Rokoko Studio
    \item Vicon Nexus
    \item MotionBuilder
\end{itemize}

\paragraph{Software de Scanning 3D}
\begin{itemize}
    \item Agisoft Metashape
    \item RealityCapture
    \item Autodesk ReCap
\end{itemize}

\paragraph{Sistema de Gerenciamento de Projetos}
\begin{itemize}
    \item Trello
    \item Asana
    \item Monday.com
\end{itemize}

\paragraph{Sistema de Streaming de Arquivos}
\begin{itemize}
    \item Google Cloud Storage
    \item Amazon S3
    \item Microsoft Azure Blob Storage
\end{itemize}


\subsection{Orçamento Estimado}

\subsubsection{Hardware}

\begin{itemize}
    \item \textbf{Estações de Edição:}
    \begin{itemize}
        \item 12 computadores equipados com processadores AMD Ryzen 9, 64GB de RAM, placas gráficas NVIDIA RTX 4090.
        \item Estimativa de custo por unidade: \$4,000
        \item Total: \$48,000
    \end{itemize}
    
    \item \textbf{Servidores de Armazenamento:}
    \begin{itemize}
        \item Servidor central com dois processadores AMD EPYC 7763, 1TB de RAM, 1PB de SSD NVMe, duas GPUs NVIDIA A100.
        \item Estimativa de custo total: \$1,200,000
    \end{itemize}
    
    \item \textbf{Equipamentos de Motion Capture:}
    \begin{itemize}
        \item 8 roupas Rokoko, 25 base stations, múltiplos microfones.
        \item Estimativa de custo por roupa Rokoko: \$6,000
        \item Estimativa de custo por base station: \$1,000
        \item Estimativa de custo total: \$128,000
    \end{itemize}
    
    \item \textbf{Equipamentos de Scanning 3D:}
    \begin{itemize}
        \item 100 câmeras Intel RealSense L515.
        \item Estimativa de custo por unidade: \$350
        \item Total: \$35,000
    \end{itemize}
    
    \item \textbf{Notebooks para Visualização:}
    \begin{itemize}
        \item Múltiplos notebooks Microsoft Surface Laptop Studio.
        \item Estimativa de custo por unidade: \$2,500
        \item Total: \$10,000 (supondo 4 unidades)
    \end{itemize}
    
    \item \textbf{Monitores e Sistema de Som:}
    \begin{itemize}
        \item Dois monitores grandes e sistema de som surround.
        \item Estimativa de custo total: \$5,000
    \end{itemize}
    
    \item \textbf{Tela Infinita de LED:}
    \begin{itemize}
        \item Painel de LED de 20m de diâmetro e 5m de altura com teto de LED integrado.
        \item Estimativa de custo total: \$500,000
    \end{itemize}
    
    \item \textbf{Câmeras Virtuais:}
    \begin{itemize}
        \item Duas câmeras virtuais NVIDIA Omniverse™ 3D Cameras.
        \item Estimativa de custo por unidade: \$10,000
        \item Total: \$20,000
    \end{itemize}
    
    \item \textbf{Equipamento de Som:}
    \begin{itemize}
        \item Sistema de microfones Sennheiser MKE 2 e fones de ouvido Sony MDR-7506.
        \item Estimativa de custo total: \$2,000
    \end{itemize}
\end{itemize}

\subsubsection{Computadores}

\begin{itemize}
    \item \textbf{Computadores na Sala de Gravação:}
    \begin{itemize}
        \item Computador de Linha: AMD Ryzen Threadripper, 128GB RAM, NVIDIA RTX 4090, 4TB SSD.
        \item Estimativa de custo: \$7,000
        \item Computador Auxiliar: Similar, com 64GB RAM.
        \item Estimativa de custo: \$5,000
        \item Total: \$12,000 (supondo 1 de linha e 1 auxiliar)
    \end{itemize}
    
    \item \textbf{Computadores no Centro de Motion Capture:}
    \begin{itemize}
        \item Computador de Linha: AMD Ryzen Threadripper, 256GB RAM, NVIDIA RTX 4090, 4TB SSD.
        \item Estimativa de custo: \$8,000
        \item Computador Auxiliar: Similar, com 128GB RAM.
        \item Estimativa de custo: \$6,000
        \item Total: \$14,000 (supondo 1 de linha e 1 auxiliar)
    \end{itemize}
    
    \item \textbf{Computadores no Centro de Scanning:}
    \begin{itemize}
        \item Computador de Linha: AMD EPYC, 256GB RAM, NVIDIA Quadro RTX 6000, 4TB SSD.
        \item Estimativa de custo: \$10,000
        \item Computador Auxiliar: AMD Ryzen 9, 128GB RAM, NVIDIA RTX 3080, 2TB SSD.
        \item Estimativa de custo: \$6,000
        \item Total: \$16,000 (supondo 1 de linha e 1 auxiliar)
    \end{itemize}
\end{itemize}

\subsubsection{Software}

\begin{itemize}
    \item \textbf{Software de Edição:}
    \begin{itemize}
        \item Adobe Premiere Pro, DaVinci Resolve.
        \item Estimativa de custo por licença anual: \$600
        \item Total: \$600
    \end{itemize}
    
    \item \textbf{Software de Motion Capture:}
    \begin{itemize}
        \item Rokoko Studio.
        \item Estimativa de custo por licença anual: \$2,000
        \item Total: \$2,000
    \end{itemize}
    
    \item \textbf{Software de Scanning 3D:}
    \begin{itemize}
        \item Agisoft Metashape.
        \item Estimativa de custo por licença: \$3,500
        \item Total: \$3,500
    \end{itemize}
    
    \item \textbf{Sistema de Gerenciamento de Projetos:}
    \begin{itemize}
        \item Estimativa de custo: \$1,000
    \end{itemize}
    
    \item \textbf{Sistema de Streaming de Arquivos:}
    \begin{itemize}
        \item Estimativa de custo: \$2,000
    \end{itemize}
\end{itemize}

\subsubsection{Banco de Dados}

\begin{itemize}
    \item \textbf{Banco de Dados de Projetos Ativos, Backup e do Site:}
    \begin{itemize}
        \item Estimativa de custo para configuração e manutenção: \$10,000
    \end{itemize}
\end{itemize}

\subsubsection{Total Estimado do Orçamento}

\begin{itemize}
    \item \textbf{Hardware:} \$1,990,000
    \item \textbf{Banco de Dados:} \$10,000
    \item \textbf{Software:} \$9,100
\end{itemize}

\textbf{Total Geral (sem mão de obra):} \$2,000,000

\textbf{Total Licenças Recorrentes (Anual):} \$9,100

\subsection{Resumo e Recomendações}

Considerando a análise das viabilidades o sistema a ser desenvolvido é \textbf{viável} do ponto de vista \textbf{técnico, organizacional e econômico}.

As recomendações finais são baseadas na análise detalhada de todos os aspectos do projeto, garantindo que todas as necessidades sejam atendidas e que o sistema atenda às expectativas e requisitos estabelecidos. Recomenda-se proceder com o desenvolvimento conforme o plano estabelecido, com atenção especial aos seguintes pontos:
\begin{itemize}
    \item \textbf{Validação Técnica}: Continuar com a avaliação contínua das tecnologias e ferramentas selecionadas para assegurar que atendem aos requisitos técnicos do projeto.
    \item \textbf{Preparação Organizacional}: Garantir que a equipe esteja bem treinada e que os processos organizacionais estejam alinhados para a implementação eficiente do sistema.
    \item \textbf{Gestão Econômica}: Monitorar os custos e ajustes financeiros durante o desenvolvimento para assegurar que o orçamento seja mantido e que os recursos sejam utilizados de maneira eficaz.
    \item \textbf{Revisão de Alternativas}: Revisar periodicamente as alternativas tecnológicas e considerar ajustes conforme necessário para melhorar a eficiência e a eficácia do sistema.
\end{itemize}
