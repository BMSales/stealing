% Prof. Dr. Ausberto S. Castro Vera
% UENF - CCT - LCMAT - Curso de Ci\^{e}ncia da Computa\c{c}\~{a}o
% Campos, RJ,  2023
% Disciplina: Paradigmas de Linguagens de Programa\c{c}\~{a}o
% Aluno: Eric Hoffmann Fernandes Braga


\chapterimage{ScalaH} % Chapter heading image ==> Trocar este arquivo por outro 1200x468
\chapter{ Programa\c{c}\~{a}o em Rust}

Nesse capítulo veremos como programar em Rust usando seu conjunto básico de funcionalidade.
\begin{itemize}
  \item Entrada e Saida
  \item Controle de Fluxo
  \item La\c{c}os
  \item Fun\c{c}oes
  \item Modulos
\end{itemize}

Ao final entenderemos sobre o processo de entrada e saída de rust, como controlar o fluxo de código com condicionais e switchs e como criar repetições como abstrair código com funções e módulos.

%%%%%%%%======================
\section{Entradas e sa\'{\i}das}
%%%%%%%%======================

Entrada e saida em Rust são gerenciadas por uma série de Macros definidas em std::fmt algumas dessas macros incluem:
\begin{itemize}
  \item \textit{format!} escreve texto formatado para \textit{String} resultado
  \item \textit{print!} faz o mesmo que \textit{format!} so que escreve no console
  \item \textit{println!} faz o mesmo que \textit{print!} so que com uma nova linha ao final
  \item \textit{eprint!} faz o mesmo que print mas escreve em erro padrao (io::stderr)
  \item \textit{eprintln!} faz o mesmo que \textit{eprint!} mas com uma nova linha ao final
\end{itemize}
\par
Para podermos usar esses métodos de entrada e saída precisamos primeiro entender como a interpolação de string funciona:

\pagebreak
\newpage

\subsection{Interpola\c{c}ao}

Para interpolar em uma \textit{String} usamos chaves '{}' dentro da string seguido do paramêtro que queremos substituir fora da \textit{String} por exemplo:

\begin{lstlisting}[language=rust]
interpolation.rs
fn main() {
  print("Ola {} numero {}", "terra", 4);
}
\end{lstlisting}

Neste exemplo podemos ver que as chaves pegam em ordem os argumentos, assim o primeiro par de chaves pega "terra" e o segundo pega "4".

Caso voce deseje tambem e possível nomear as váriaveis que você deseja dentro das chaves para pegá-las em quaisquer ordem através de seu nome.

\begin{lstlisting}[language=rust]
interpolation.rs
fn main() {
  print("Ola {planeta} numero {numero}", planeta = "terra", numero = 4);
}
\end{lstlisting}

Outra maneira de interpolar em strings é numerar as chaves com qual váriavel você deseja usar:

\begin{lstlisting}[language=rust]
interpolation.rs
fn main() {
  print("{0} gosta de {1} mas {1} nao gosta de {0}", "Joao", "Alice");
}
\end{lstlisting}

Dentro da chaves tambem é possivel formatar os dados sendo interpolados:

\begin{lstlisting}[language=rust]
interpolation.rs
fn main() {
  println!("Base 10: {}", 69420); // 69420
  println!("Base 2 (binary): {:b}", 69420); // 10000111100101100
  println!("Base 8 (octal): {:o}", 69420); // 207454
  println!("Base 16 (hexadecimal): {:x}", 69420); // 10f2c
}
\end{lstlisting}

Tambem é possivel justificar texto:

\begin{lstlisting}[language=rust]
interpolation.rs
fn main() {
  println!("{number:>5}", number=1); // "    1"
  println!("{number:<5}", number=1); // "1    "
  // Voce pode substituir o caractere do padding pelo caracter que quiser:
  println!("{number:0>5}", number=1); // "00001"
  println!("{number:0<5}", number=1); // "10000"
  // e possivel usar argumentos nomeados com "$"
  println!("{number:0>$width}", number = 1, width = 5); // "00001"
}
\end{lstlisting}

\subsection{Format!}

\textit{Format!} recebe um string e váriaveis interpola as váriaveis na string e retorna como outra string para ser usada no código.


\begin{lstlisting}[language=rust]
format.rs
fn main() {
  let x = format!("Ola {}!", "Mundo");
  print!(x); // Resultado: "Ola Mundo!"
}
\end{lstlisting}

\subsection{Print!}

\textit{Print!} faz o mesmo que \textit{Format!} só que escreve o resultado em std::out (console/terminal) 

\begin{lstlisting}[language=rust]
print.rs
fn main() {
  print!("Ola {}!", "Mundo");
  print!("Ola {}!", "Mundo"); 
  // Resultado: "Ola Mundo!Ola Mundo!"
}
\end{lstlisting}

\subsection{Println!}

\textit{Println!} faz o mesmo que \textit{Print!} só que acrescenta uma nova linha ao final.

\begin{lstlisting}[language=rust]
println.rs
fn main() {
  println!("Ola {}!", "Mundo");
  println!("Ola {}!", "Mundo");
  // Resultado: "Ola Mundo!
  //             Ola Mundo!"
}
\end{lstlisting}

\subsection{Saidas de Erros}

\textit{Eprint! e Eprintln!} funcionam igual a \textit{Print! e Println!} só que sua saída de texto e em io:stderr, isso significa que essas mensagens ao ocorrer fecharão seu programa com base em uma condição e mostrarão essa mensagem no cosole apos o programa ter fechado.

\begin{lstlisting}[language=rust]
errors.rs
fn main() {
  let config = Config::build(&args).unwrap_or_else(|err| {
    eprintln!("Problem parsing arguments: {err}");
    process::exit(1);
  });}
\end{lstlisting}

Se tentarmos rodar esse programa com argumentos insuficientes teremos:

\begin{lstlisting}[language=bash]
$ cargo run > out.txt
> Problem parsing arguments: not enough arguments
\end{lstlisting}

%%%%%%%%======================
\section{Controle de Fluxo}
%%%%%%%%======================

Temos múltiplas maneiras de controlar quais pedaços de codigo queremos executar e quantas vezes queremos executar, pra isso se da o nome de controle de fluxo, pois estamos controlando por onde e como nosso programa passa e executa, algumas estruturas que usamos para isso são:

\subsection{If e If else}

O \textit{if} deixa você avaliar expressões lógicas e com base em seu resultado executar trechos específicos de código:

\begin{lstlisting}[language=rust]
if.rs
fn main() {
  let x = 5;
  if x < 6 {
    println!("X e menor que 6");
  } else if x == 6{
    println!("X e 6");
  } else {
    println!("X e maior que 6");
  }
\end{lstlisting}

Aqui criamos uma condição lógica onde se x for menor que 5 temos o resultado: "X e menor que 6" que é a primeira condicao do if e so acontece caso a expressao logica seja verdadeira, ao mudar x para 6 caimos no segundo caso onde olhamos outras possibilidades e avaliamos elas novamente com outro if atraves da concatenacao de um "else" e um "if", entao conseguimos o resultado "X e 6", e por ultimo caso mudemos x para 7 cairemos no caso else que so acontece caso todas as expressoes logicas anteriores sejam falsas, e teremos o resultado: "X e maior q 6". 

\subsection{Match}

A estrutura match e extramente semelhante a estrutura de \textit{Switch Case} em C, onde podemos especificar varios casos e usar casamento de padroes em cima de uma variavel alvo para decidir qual caso executar:

\begin{lstlisting}[language=rust]
match.rs
fn main() {
  let x = 12;
  match number {
    // Podemos casar o padrao com um unico valor
    1 => println!("One!");
    // Com multiplos valores
    2 | 3 | 5 | 7 | 11 => println!("Numero primo!");
    // Com intervalos inclusivos
    13..=19 => println!("Um Adolescente!");
    // Lidar com quaisquer outro caso possivel
    // que nao tenha sido especificado
    _ => println!("Nada de especial");
  }
}
\end{lstlisting}

\subsection{La\c{c}os de Repeti\c{c}\~{a}o}

Estruturas de laco de repeticao nos permitem repetir trechos de codigo de diversas maneiras, desde enquanto uma expressao logica for verdade ate infinitamente:
\par
O primeiro e mais simples laco que existe e o \textit{loop}, uma estrutura que executa seu escopo infinitamente a nao ser que encontre um \textit{break}.
\begin{lstlisting}[language=rust]
loops.rs
fn main() {
  loop {
    println!("Ola Mundo!");
  }
  \\ Resultado:
  \\ "Ola Mundo!"
  \\ "Ola Mundo!"
  \\ "..."
  \\ Executara infinitamente
}
\end{lstlisting}
O laco \textit{while} pode ser usado para executar um trecho de codigo ate que uma condicao se torne falsa
\begin{lstlisting}[language=rust]
loops.rs
fn main() {
  let mut x = 1;
  while x < 10 {
    println!("Ola Mundo!");
    x += 1;
  }
  \\ Resultado: Escrevera "Ola Mundo!" dez vezes
}
\end{lstlisting}
O ultimo laco o \textit{for range} permite que seja extraido um dado de uma estrutura e o use durante o laco que esta sendo executado:
\begin{lstlisting}[language=rust]
loops.rs
fn main() {
  let names = vec!["Bob", "Frank", "Ferris"];
  for x in 1..=100 { println("{}", x); }
  \\ Resulta em todos os numeros de 1 a 100 sendo escritos
  \\ Aqui vemos duas estruturas de controle de fluxo 
  \\ sendo usadas em paralelo
  for name in names.iter() {
    match name {
      &"Ferris" => println!("Tem um Rustaceano entre nos"),
      _ => println!("Ola {}", name),
    }
  }
  \\ Resultado: caso ele ache o nome ferris no vetor ele escrevera
  \\ "Tem um Rustaceano entre nos" e "Ola Ferris"
  \\ Caso o nome nao esteja no vetor nada sera escrito
}
\end{lstlisting}

%%%%%%%%======================
\section{Fun\c{c}\~{o}es}
%%%%%%%%======================

Funcoes em Rust sao declaradas usando a palavra chave "fn", seus valores de entrada devem ser \textit{type-annotated} e se a funcao possui valor de retorno ele deve ser indicado usando "->"
\par
A ultima expressao dentro de uma funcao sera usada como valor de retorno, alternativamente a palavra chave \textit{return} pode ser usada para retornar um valor antes da ultima expressao, mesmo de dentro de lacos de repeticao ou estruturas logicas.
\par Diferente de C nao ha restricao na ordem da definicao de funcoes, lhe permitindo chamar uma funcao no codigo, em linhas anteriores a sua declaracao.
\\
\\
{\huge Exemplo:}
\begin{lstlisting}[language=rust]
functions.rs
fn main() {
  // Chamada antes da declaracao e possivel  
  if is_odd(3) {
    println!("Impar");
  } else {
    println!("Par");
  }
  hello("Sophia");
}

fn is_odd(n: i32) -> bool {
  if n % 2 == 0 { return false; }
  // Para explicitar ultima expressao nao se inclui ";"
  n % 2 == 0
}

// Quando uma funcao nao retorna nada ela na verdade retorna "()"
// Quando a funcao retorna o tipo unitario "()"
// O retorno pode ser omitido
fn hello(x: str) { println!("Hello {}!", x); }
\end{lstlisting}

\pagebreak
\newpage

%%%%%%%%======================
\section{M\'{o}dulos}
%%%%%%%%======================

Modulos sao uma colecao de itens como funcoes, structs, \textit{traits}, \textit{impl} onde voce pode acessar todos seus elementos.

\begin{lstlisting}[language=rust]
functions.rs
mod meu_modulo{
  // Todas as funcoes em modulos sao privadas por padrao
  fn helloMod() {
      println!("Hello Module!");
  }
  // A funcao pode ser tornada publica pela palavra chave 'pub'
  pub fn helloWorld() {
      println!("Hello World");
  }
  // Itens dentro de um modulo podem ser acessados por outros itens
  // Modulos podem ser aninhados 
  //
}

fn main() {
  // Se accessa itens dentro de modulos usando '::'
  meu_modulo::helloWorld();
}
\end{lstlisting}

